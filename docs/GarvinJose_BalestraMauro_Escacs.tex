\documentclass[12pt]{article}
\usepackage{comment}
\usepackage{listings}
\usepackage{enumitem}
\usepackage[utf8]{inputenc}
\usepackage[catalan]{babel}
\usepackage{graphicx}
\usepackage{wrapfig}
\usepackage{booktabs}
\usepackage{hyperref}
\hypersetup{
    colorlinks=true,
    linkcolor=black,
    filecolor=magenta,      
    urlcolor=cyan,
}

\graphicspath{ {images/} }
\usepackage{geometry}
 \geometry{
 a4paper,
 total={170mm,257mm},
 left=25mm,
 top=25mm,
 bottom=25mm
 }
 
 \usepackage[most]{tcolorbox}
\newtcolorbox{myframe}[1][]{
  enhanced,
  arc=0pt,
  outer arc=0pt,
  colback=white,
  boxrule=0.8pt,
  #1
}
 


\begin{document}

\begin{titlepage}

\begin{center}
\vspace*{1in}
\begin{figure}[htb]
\begin{center}
\includegraphics[width=5cm]{logo}
\end{center}
\end{figure}

PARADIGMES I LLENGUATGES DE PROGRAMACIÓ
\vspace*{0.15in}
\vspace*{0.6in}
\begin{large}
\\
\end{large}
\vspace*{0.2in}
\begin{Large}
\textbf{PRÀCTICA DE HASKELL} \\
\end{Large}
\vspace*{0.2in}
\begin{large}
Validador de jugades d'escacs\\
\end{large}
\vspace*{0.2in}
\rule{80mm}{0.1mm}\\
\vspace*{0.2in}
\begin{large}
Jose Garvin Victoria \\
Mauro Balestra Sastriques \\
\vspace*{0.2in}
 \textbf{Enginyeria informàtica}\\
 \vspace*{0.2in}
 \textbf{Curs 2018/19}
\end{large}
\end{center}

\end{titlepage}



\newpage
\tableofcontents

\newpage
\section{Introducció i abast}	

\section{Resolució del problema}

\section{Definicions de tipus}
\subsection{Funcions a destacar}

\section{Execució i proves}
\subsection{Com executar}
\subsection{Jocs de proves}

\section{Observacions}

\newpage
\section{Bibliografía}

\setitemize{labelindent=-2em,labelsep=1cm,leftmargin=*}
\begin{itemize}[label={}]

\item LXC, Wikipedia, (2018), Recuperat el 2 d'abril de 2018 de \url{https://en.wikipedia.org/wiki/LXC}

\end{itemize}

\end{document}
